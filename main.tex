% ****** Start of file apssamp.tex ******
%
%   This file is part of the APS files in the REVTeX 4.1 distribution.
%   Version 4.1r of REVTeX, August 2010
%
%   Copyright (c) 2009, 2010 The American Physical Society.
%
%   See the REVTeX 4 README file for restrictions and more information.
%
% TeX'ing this file requires that you have AMS-LaTeX 2.0 installed
% as well as the rest of the prerequisites for REVTeX 4.1
%
% See the REVTeX 4 README file
% It also requires running BibTeX. The commands are as follows:
%
%  1)  latex apssamp.tex
%  2)  bibtex apssamp
%  3)  latex apssamp.tex
%  4)  latex apssamp.tex
%
\documentclass[%
 preprint,
%superscriptaddress,
%groupedaddress,
%unsortedaddress, 
%runinaddress,
%frontmatterverbose, 
%preprint,
%showpacs,preprintnumbers,
%nofootinbib,
%nobibnotes,
%bibnotes,
 amsmath,amssymb,
 aps,
%pra,
%prb,
%rmp,
%prstab,
%prstper,
%floatfix,
]{revtex4-1}

\usepackage{graphicx}% Include figure files
\usepackage{dcolumn}% Align table columns on decimal point
\usepackage{bm}% bold math
%\usepackage{hyperref}% add hypertext capabilities
%\usepackage[mathlines]{lineno}% Enable numbering of text and display math
%\linenumbers\relax % Commence numbering lines

%\usepackage[showframe,%Uncomment any one of the following lines to test 
%%scale=0.7, marginratio={1:1, 2:3}, ignoreall,% default settings
%%text={7in,10in},centering,
%%margin=1.5in,
%%total={6.5in,8.75in}, top=1.2in, left=0.9in, includefoot,
%%height=10in,a5paper,hmargin={3cm,0.8in},
%]{geometry}

\usepackage{braket} % Allows for Dirac notation
\usepackage{subcaption}

\newcommand{\veff}{\hat{V}_{12,\text{eff}}}
\newcommand{\rhozero}{\rho_{\text{I}=0}}
\newcommand{\rhoone}{\rho_{\text{I}=1}}

\newcommand{\rhohat}[2]{\hat{\rho}_{\text{I}=#1}\left( #2 \right)}

\newcommand{\yukawa}[1]{\frac{e^{-m_\pi |#1|}}{4\pi |#1|}}
\newcommand{\yukawadimless}[1]{\frac{e^{-m_\pi #1}}{m_\pi #1}}
\newcommand{\yukawanoabs}[1]{\frac{e^{-m_\pi #1}}{4\pi #1}}

\newcommand{\rot}{\vec{r}_{12}}
\newcommand{\rotp}{\vec{r}_{12}\!\!\!'\,\,}

\newcommand{\rotpr}{r_{12}\!\!\!\!'\,\,\,}
\newcommand{\rotphat}{\hat{r}_{12}\!\!\!\!'\,\,\,}

\newcommand{\taudot}{\vec{\tau}_1\cdot\vec{\tau}_2}
\newcommand{\taucrossthree}{\left[\vec{\tau}_1\times\vec{\tau}_2\right]_3}
\newcommand{\tauplusthree}{\frac{\tau_1^3+\tau_2^3}{2}}
\newcommand{\tauminusthree}{\frac{\tau_1^3+\tau_2^3}{2}}

\newcommand{\sigmadot}{\vec{\sigma}_1\cdot\vec{\sigma}_2}
\newcommand{\sigmaplus}{\vec{\sigma}_1+\vec{\sigma}_2}
\newcommand{\sigmaminus}{\vec{\sigma}_1-\vec{\sigma}_2}
\newcommand{\sigmatwo}{[\vec{\sigma}_1\otimes\vec{\sigma}_2]_2}
\newcommand{\sigmaone}{[\vec{\sigma}_1\otimes\vec{\sigma}_2]_1}
\newcommand{\sigmacross}{\vec{\sigma}_1\times\vec{\sigma}_2}

\newcommand{\fracphantom}{\vphantom{\frac{1}{1}}}

\newcommand{\lvec}[1]{\reflectbox{\ensuremath{\vec{\reflectbox{\ensuremath{#1}}}}}}

%\newcommand{\w}[4]{w(#1,#2,#3;#4)}
\newcommand{\w}[4]{w_{#1,#2,#3}(#4)}

\begin{document}

\preprint{APS/123-QED}

\title{An Analytic Reduction of the Chiral Three-Nucleon Potential to an In-Medium Two-Nucleon Form}% Force line breaks with \\
%\thanks{A footnote to the article title}%

\author{Cory D. Schillaci}
\email{schillaci@berkeley.edu}
%\author{Mark C. Strother}
%\email{mcstro@berkeley.edu}
 %\altaffiliation[Also at ]{Physics Department, XYZ University.}%Lines break automatically or can be forced with \\
\author{Wick C. Haxton}%
\email{haxton@berkeley.edu}
\affiliation{%
Department of Physics, University of California, Berkeley, California 94720, USA
}%

\date{\today}% It is always \today, today,
             %  but any date may be explicitly specified

\begin{abstract}
We analytically reduce the chiral three-nucleon interaction at N$^2$LO to a density-dependent effective two-body potential by summing the third particle over the states of a spin-symmetric Fermi gas. Results are given for the potential in momentum space and in coordinate space. A local expansion of the coordinate space potential is also derived. 
\end{abstract}

%\pacs{Valid PACS appear here}% PACS, the Physics and Astronomy
                             % Classification Scheme.
%\keywords{Suggested keywords}%Use showkeys class option if keyword
                              %display desired
\maketitle

%\tableofcontents

\section{\label{sec:level1}Introduction}

Three-nucleon (3N) interactions first appear in the chiral lagrangian at order N$^2$LO. Numerous studies have shown that inclusion of the 3N forces is essential for correctly modeling nuclear physics in all regimes, from light and medium nuclei \cite{PhysRevLett.99.042501,0954-3899-39-8-085111} to nuclear matter \cite{PhysRevC.82.014314,PhysRevC.83.031301}.

However, in modern numerical approaches such as the ab initio no core shell model, including 3N forces exactly in calculations requires the use of basis spaces which are orders of magnitude larger than the two-nucleon (2N) case. The increased demands on memory and computing hours rapidly become prohibitive even for light nuclei \cite{Barrett2013131}. 

Because we have a highly accurate but computationally intensive theory, this suggests an effective theory approach. Here, we analytically reduce the chiral three-body interaction to an average two-body interaction which depends on the local nucleon density. This approach has been successfuly applied in the reduction of two-body interactions to single particle potentials \cite{PhysRev.133.B329,AdelbergerHaxton}. 

Early efforts to develop a two-body effective interaction for the 3N part of the chiral potential focussed  on cases with specific isospin constraints. Explicit expressions for  the effective potential in momentum space have been derived for pure neutron matter \cite{PhysRevC.82.014314} and for isospin symmetric nuclei \cite{PhysRevC.81.024002}. These results have been applied succesfully in calculations of nuclear pairing energies using nuclear energy density functionals \cite{0954-3899-39-1-015108}. The momentum space potential for arbitrary isospins was recently derived using \texttt{MATHEMATICA} for asymmetric isospins, although no explicit expressions were given\cite{Drischler:2015eba}. An alternate approach to deriving density-dependent effective potential using correlated basis functions was also proposed in \cite{PhysRevC.83.054003}.

Here, we derive and state expressions for the effective potential valid for arbitrary isospin composition and without neglecting any contributions. The averaging procedure to derive the effective potential is described in Section~\ref{sec:averaging}. Expressions for the effective potential are given in both momentum and coordinate space (Sections~\ref{sec:momentum} and \ref{sec:coord}, respectively). All coordinate space potentials are written fully in terms of spherical tensors in order to facilitate their use. In Section~\ref{sec:local}, we perform an expansion to obtain a local approximation for the full effective potential, which otherwise contains nonlocal terms.

\section{\label{sec:averaging}Averaging over a Fermi Gas core}

We can imagine a nucleus such as $^{18}$F or $^{42}$Ca in which there are two particles outside an inert core. These particles interact via the 3N force with all of the core nucleons as well. If we model the core as a spin-symmetric Fermi gas, then summation over the interactions with the core nucleons gives a dependence on the Fermi momentum, which is analytically related to the density of the core. 

The Fermi gas states are momentum eigenstates of the form
\begin{equation}
\ket{\alpha}=\ket{\vec{k}_\alpha;m_{s_\alpha},m_{t_\alpha}}
\end{equation}
where $m_s$ and $m_t$ are the spin and isospin projections, respectively. %We normalize the momentum eigenstates such that $\braket{\vec{k} | \vec{k}'}=(2\pi)^3\delta^{(3)}(\vec{k}-\vec{k}')$. 

We can then sum over interactions with one core nucleon to generate an effective two-body potential $V_{12,\text{eff}}$ such that
%\begin{equation}
%\bra{\alpha_1,\alpha_2}\overline{V}_{3N}\ket{\beta_1,\beta_2}=\sum_{\gamma}\braket{\alpha_1,\alpha_2,\gamma | V_{3N} | \beta_1,\beta_2,\gamma}
%\end{equation}
\begin{equation}
\frac{1}{2} \bra{\alpha_1\:\alpha_2}  V_{12,\text{eff}}\ket{\beta_1 \:\beta_2 }=\frac{1}{6} \sum_{\gamma} \bra{\alpha_1\:\alpha_2 \: \gamma} \hat{V}_{123}  \ket{\beta_1 \:\beta_2 \gamma}
\end{equation}
where the the Fermi gas is assumed to be spin symmetric but not necessarily isospin symmetric, so that the sum can be expanded as
\begin{equation}
\sum_\gamma=\sum_{m_s(\gamma)=\pm 1/2}\int \frac{d^3k_\gamma}{(2\pi)^3} \Big[\delta_{m_{t}(\gamma),+1/2} n_p(k_\delta) +\delta_{m_{t}(\gamma),-1/2} n_p(k_\delta) \Big]
\end{equation}
with $n_p(k)$ and $n_n(k)$ the density of states for the protons and neutrons in the fermi gas, respectively. We will use the $T=0$ Fermi-Dirac distribution, $n_p(k)=\theta(k_{F,p}-k)$ and $n_n(k)=\theta(k_{F,n}-k)$. However, if another density of states would be more appropriate the effect would merely be to change the integrals \eqref{eq:kF}(GIVE EQN NUMBERS).

Density dependence arises from the momentum integrals. The standard relationship for a homogeneous Fermi gas with two internal spin degrees of freedom is 
\begin{equation}\label{eq:kF}
\rho=\frac{1}{V}\sum_{m_{s}=\pm 1/2}\int \frac{d^3\vec{k}_\delta}{(2\pi)^3}\:n(k_\delta)
=\frac{1}{V}\sum_{m_{s}=\pm 1/2}\int^{|k_\delta|<k_F}\frac{d^3\vec{k}_\delta}{(2\pi)^3}=\frac{k_F^3}{3\pi^2}
\end{equation}

As they will arise repeatedly in our calculations, we define dimensionless isoscalar and isovector combinations of the densities:
\begin{equation}\label{eq:densities}
\rho_{I=0,1}=\frac{\rho_P\pm\rho_N}{m_\pi^3}
\end{equation}

Because nucleons obey Fermi statistics, the nuclear Hamiltonian must commute with the antisymmetrization operator $\mathcal{A}_{123}$. Alternately, we can instead obtain the same matrix elements by evaluating the potential between antisymmetric states, e.g.
\begin{equation}
\begin{split}
\ket{\alpha_1\:\alpha_2\:\alpha_3}_{\text{assym}} = \sqrt{\frac{1}{6}} \left( \ket{\alpha_1\:\alpha_2\:\alpha_3}\right. & - \ket{\alpha_1\:\alpha_3\:\alpha_2} - \ket{\alpha_2\:\alpha_1\:\alpha_3}   \\
&\hspace{.5cm}\left. +\ket{\alpha_2\:\alpha_3\:\alpha_1} - \ket{\alpha_3\:\alpha_2\:\alpha_1} + \ket{\alpha_3\:\alpha_1\:\alpha_2} \right)
\end{split}
\end{equation}
Throughout this paper we will, for brevity, generally write the potential in a form which is not fully antisymmetrized but which is equivalent to the correct potential when evaluated between antisymmetric states. 

\section{\label{sec:momentum} The effective potential in momentum space}

\begin{figure}
\centering
\begin{subfigure}{0.25\textwidth}
\includegraphics[page=4]{Figures/3NFDiagrams}
%\caption{\label{subfig:3NC}}
\end{subfigure}
\begin{subfigure}{0.25\textwidth}
\includegraphics[page=3]{Figures/3NFDiagrams}
%\caption{}
\end{subfigure}
\begin{subfigure}{0.25\textwidth}
\includegraphics[page=2]{Figures/3NFDiagrams}
%\caption{}
\end{subfigure}
\caption{\label{fig:3NF}Diagrams for the 3N interactions at NNLO.}
\end{figure}

There are three three-body terms in the chiral potential at NNLO \cite{PhysRevC.66.064001}. 
\begin{align}
V_E&=\frac{1}{2}\frac{c_E }{F_\pi^4\Lambda_\chi}\sum_{i\neq j} \vec{\tau}_i\cdot\vec{\tau}_j \label{eq:V_E} \\
V_D&=-\frac{ g_A}{8F_\pi^2}\frac{c_D}{\Lambda_\chi F_\pi^2}\sum_{i\neq j \neq k} \frac{ \vec{\sigma}_i\cdot\vec{q}_j\:\vec{\sigma}_j\cdot\vec{q}_j }{q^2_j+m_\pi^2} \vec{\tau}_i\cdot\vec{\tau}_j \label{eq:V_D}\\
V_{C} &= \frac{1}{2}\left(\frac{g_A}{2F_\pi}\right)^2\sum_{i\neq j \neq k} \frac{ \vec{\sigma}_i\cdot\vec{q}_i}{q_i^2+m_\pi^2}\frac{\vec{\sigma}_j\cdot\vec{q}_j }{q^2_j+m_\pi^2} F_{ijk}^{\alpha\beta}\tau_i^{\alpha}\tau_j^\beta \label{eq:V_c}
\end{align}

where 
\begin{equation}
F_{ijk}^{\alpha\beta}=\delta^{\alpha \beta}\left[-\frac{4c_1m_\pi^2}{F_\pi^2}+\frac{2c_3}{F_\pi^2}\vec{q}_i\cdot\vec{q}_j\right]+\sum_\gamma\frac{c_4}{F_\pi^2}\epsilon^{\alpha\beta\gamma}\tau^\gamma_k\vec{\sigma}_k\cdot\left(\vec{q}_i\times\vec{q}_j\right)
\end{equation}
and $\vec{q}_i=\vec{k}_i' - \vec{k}$ is the difference in the final and initial state momenta for particle $i \in \{1,2,3\}$.

These represent a three-body contact potential, a one-pion exchange plus contact interaction (1PE), and a two-pion exchange (2PE) interaction as shown in Figure \ref{fig:3NF}. Note that the 2PE term can be split into parts proportional to $c_1, c_3$ and $c_4$ as $V_C=V_1+V_2+V_4$. Analytically summing over the Fermi gas particles to find an effective potential corresponds to the summations shown in Figure \ref{fig:eff-diagram}. Note that $c_D$ and $c_E$ are unitless, while $c_1, c_3$ and $c_4$ have units of inverse energy. 

\begin{figure}
\includegraphics[scale=0.6,page=1]{Figures/InMediumDiagrams2}
\caption{\label{fig:eff-diagram} Diagrammatic representation of $V_{12,\text{eff}}$.}
\end{figure}

For a standard two-body interaction, the matrix elements may depend only on the quantities
 \begin{equation}
 \vec{k}=\frac{\vec{k}_1-\vec{k}_2}{2},\quad \vec{k}'=\frac{\vec{k}'_1-\vec{k}'_2}{2}
 \end{equation}
 in order for the Hamiltonian to be Galilean invariant. However, the original three-body interaction may also depend on the total momentum of the two particles,
 \begin{equation}
 \vec{P}=\vec{k}_1+\vec{k}_2, \quad  \vec{P}'=\vec{k}'_1+\vec{k}'_2
 \end{equation}
 through the relative Jacobi momenta 
 \begin{equation}
 \vec{\pi}_1=\frac{1}{\sqrt{2}}\left(\vec{k}_1-\vec{k}_2\right), \quad  \vec{\pi}_2=\frac{1}{\sqrt{6}}\left(\vec{k}_1+\vec{k}_2-2\vec{k}_3\right).
 \end{equation}
 
We will make the approximation that the net momentum of the two valence particles is zero, and therefore so is the net momentum of the Fermi gas particles. Although this makes sense in the context of a nucleus with an inert core, for full many-body calculations it cannot be true that all pairs of particles have zero total momentum in a given reference frame. For a Fermi gas, the average net momentum of two particles scales as 
$\overline{|k_1+k_2|} \sim V^2 k_F^7 
%\sim N^2 k_F 
\sim N^2 \rho^{1/3}$. 
We therefore expect that the $P=0$ approximation is valid for many-body calculations when the density is low. This density-dependence is confirmed for nuclear matter by \cite{Drischler:2015eba}, where the authors also note that it seems to be a better approximation when the system is very isospin asymmetric. 
 
 The effective interactions for \eqref{eq:V_E} and \eqref{eq:V_D} are given in momentum space by
 \begin{align}
 \veff^E=-\frac{3c_Em_\pi^3}{2F_\pi^4 \Lambda_\chi}&\left(\rhozero-\rhoone\tauplusthree\right)\\
 \begin{split}
 \veff^D=-\frac{c_Dg_A m_\pi^3}{8 F_\pi^4\Lambda_\chi}&\left[\rhozero\taudot \frac{\vec{\sigma}_1\cdot\vec{q}\:\vec{\sigma}_2\cdot\vec{q} }{q^2+m_\pi^2}  - \rhoone \tauplusthree\frac{\vec{\sigma}_1\cdot\vec{q}\:\vec{\sigma}_2\cdot\vec{q} }{q^2+m_\pi^2} \right. \\
&+ \left(3 -\taudot\vec{\sigma}_1\cdot\hat{k}\:\vec{\sigma}_2\cdot\hat{k}\, \right) \Big[\Gamma_{0,I=0}(k)-2 \Gamma_{1,I=0}(k)\Big] \\
&+  \left(3 -\taudot\frac{\sigmadot}{3} \right) \,\Gamma_{2,I=0}(k) \\
 &-\tauplusthree \left(1 +\taudot\vec{\sigma}_1\cdot\hat{k}\:\vec{\sigma}_2\cdot\hat{k} \right) \Big[\Gamma_{0,I=1}(k)-2 \Gamma_{1,I=1}(k)\Big] \\
 &-\tauplusthree  \left(1+\taudot\frac{\sigmadot}{3} \right) \Gamma_{2,I=1}(k)
+ \vec{k}\leftrightarrow\vec{k}'\left.\fracphantom\right]
 \end{split} 
 \end{align}
Note that neither of these terms contribute for systems of pure neutrons or protons due to the antisymmetrization, consistent with the behavior of the original 3N interactions which also vanish. The functions $\Gamma_{\alpha,I=0,1}(k)=\Gamma_{\alpha,P}(k)\pm\Gamma_{\alpha,N}(k)$ are momentum dependent analogues of the densities \eqref{eq:densities} which arise from integrating over the spectator momentum,
\begin{equation}
\Gamma_{\alpha,N,P}(k) = \frac{k^{2-\alpha}}{m_\pi^3}\int\frac{d^3\vec{k}_\delta}{(2\pi)^3} n_{N,P}(k_\delta) \frac{\left\{1,k_\delta\cos\theta,k_\delta^2\right\}_\alpha}{(\vec{k}-\vec{k}_\delta)^2+m_\pi^2}
\end{equation}
Explicit expressions for these sums with the Fermi-Dirac density of states is given in appendix~\ref{app:momSums}.

\begin{align}
 \veff^{c_1}=-&c_1 m_\pi^5\left(\frac{g_A}{F_\pi^2}\right)^2 \left[ \rhozero\vec{\tau}_1\cdot\vec{\tau}_2 \:\frac{\vec{\sigma}_1\cdot\vec{q}\:\vec{\sigma}_2\cdot\vec{q} }{\left(q^2+m_\pi^2\right)^2} \right]
\end{align}

DISCUSS: In \cite{PhysRevC.81.024002}, an argument is made that for the interaction in isospin symmetric matter, the effective terms generated from the 3N interaction are partially corrections to 2N 1PE. In particular, $c_D$ term reduces (eqn 24) and $c_E$ enhances (eqn 11) 
 
\section{\label{sec:coord}The effective potential in coordinate space}

By evaluating the Fourier transforms of the momentum space effective potential, we obtain the potential in coordinate space. Each term in the full potential contains couplings of varying numbers of vector operators operating on spatial and spin quantum numbers. Throughout, we give the potentials by first coupling any coordinate operators with one another (e.g. forming the spherical harmonics $Y_l(\hat{r}_{12})$ and $Y_l(\rotphat)$), then coupling these operators together before finally coupling with the spin operators.

The simplest of the three-body interactions at N$^2$LO is the contact term. Evaluation of the diagram and summation over the core particles gives the two-body effective potential,
\begin{equation}
V_{12,\text{eff}}(\vec{r}_{12})=\frac{c_E }{2F_\pi^4\Lambda_\chi}\frac{m_\pi^6}{4\pi}\:3\w{0}{0}{0}{m_\pi r_{12}}\left[-\rho_{I=0}+ \rho_{I=1}\tauplusthree \right]
\end{equation}
which is spin-independent.
%where the dimensionless isoscalar and isovector spectator nucleon densities are given by,
%\begin{equation}\label{eq:densities}
%\rho_{I=0,1}=\frac{\rho_P\pm\rho_N}{m_\pi^3}
%\end{equation}

The three-body one pion exchange term (the middle diagram in Figure \ref{fig:3NF}) generates a richer effective interaction. The momentum dependence generates both a purely local interaction
\begin{multline}
V^{D}_{12,\text{eff}}(\rot)=
%&\frac{c_D g_Am_\pi^6}{8 F_\pi^4 \Lambda_\chi}\: \left[\rho_{I=0}\left(\taudot W^{\text{LR}}_{1PE}(\rot)  +3\frac{\delta^{(3)}(\rot)}{m_\pi^3}\right) \right. \\
%&-\left. \rho_{I=1} \left(W^{\text{LR}}_{1PE}(\rot)+(1-2\sigmadot/3)\frac{\delta^{(3)}(\rot)}{m_\pi^3}\right) \frac{\tau_3(1)+\tau_3(2)}{2} \right]
\frac{c_D g_A}{8 F_\pi^4 \Lambda_\chi} \frac{m_\pi^6}{4\pi}\: \left[\rho_{I=0}\left(\fracphantom\taudot\; W^{\text{LR}}_{1PE}(\rot)  +3\;\w{0}{0}{0}{m_\pi r}\right) \right. \\
-\left. \rho_{I=1}\;\tauplusthree \;\left(\fracphantom W^{\text{LR}}_{1PE}(\rot)+(1-2\sigmadot/3)\;\w{0}{0}{0}{m_\pi r}\right) \right]
\end{multline}
and a nonlocal potential arising from the exchange terms 
\begin{multline}\label{eq:1PENonlocal}
V^{D}_{12,\text{eff}}(\rot,\rotp)=\frac{-c_D g_A }{8 F_\pi^4 \Lambda_\chi} \frac{m_\pi^6}{32\pi^2}\:  \left[ \w{0}{0}{0}{m_\pi r_{12}} \left\{\fracphantom \rhohat{0}{r_{12}}\left(\taudot W^{\text{LR}}_{1PE}(\rotp)+3\:\w{0}{1}{0}{m_\pi \rotpr}\right) \right.\right. \\
\left.\left.+\hat{\rho}_{I=1}(r'_{12}) \tauplusthree \left( W^{\text{LR}}_{1PE}(\rotp)-\w{0}{1}{0}{m_\pi \rotpr}\right) \right\} + \rot \leftrightarrow \rotp \vphantom{\left(\yukawa{r}^2\right)} \right]
\end{multline}
In order to highlight the physical meaning of this expression, we have written part of this result using a dimensionless scalar function proportional to the long range terms in the two-body one pion exchange potential, defined as
\begin{equation}\begin{split}
W^{\text{LR}}_{1PE}(\vec{r})&=\frac{e^{-m_\pi r}}{m_\pi r}\left[\left(\vec{\sigma}_1\cdot\hat{r}\:\vec{\sigma}_2\cdot\hat{r}-\sigmadot\right)\left(1+\frac{3}{m_\pi r}+\frac{3}{(m_\pi r)^2}\right)+\frac{\sigmadot}{3}\right] \\
%&=\frac{1}{4 \pi}\frac{e^{-m_\pi r}}{m_\pi r}\left[\sqrt{\frac{8\pi}{15}}Y_2(\hat{r})\cdot\sigmatwo\left(1+\frac{3}{m_\pi r}+\frac{3}{(m_\pi r)^2}\right)+\frac{\sigmadot}{3}\right] \\
&=\left[\sqrt{\frac{8\pi}{15}}Y_2(\hat{r})\cdot\sigmatwo\w{2}{1}{2}{m_\pi r}+\frac{\sigmadot}{3}\w{0}{1}{0}{m_\pi r}\right]
\end{split}
\end{equation}

The density dependence is now also mixed with spatial dependence, which we define in analogy with \eqref{eq:densities} as 
\begin{equation}\label{eq:hatdensities}
\hat{\rho}_{I=0,1}(r)=%\frac{\rho_P}{m_\pi^3}\:\frac{3 j_1(k^P_F r)}{k_F^P r}\pm\frac{\rho_N}{m_\pi^3}\:\frac{3 j_1(k^N_F r)}{k_F^N r}=
\frac{1}{m_\pi^3}\left[ \left(\frac{3 \rho_P}{\pi}\right)^{2/3} \frac{ j_1( [3\pi^2 \rho_P]^{1/3}  r)}{ r} \pm \left(\frac{3 \rho_N}{\pi}\right)^{2/3} \frac{ j_1( [3\pi^2 \rho_N]^{1/3}  r)}{ r} \right].
\end{equation}

For all pieces of the N$^2$LO 3N interaction besides the purely short range contact $V_E$, nonlocal effective interactions like \eqref{eq:1PENonlocal} are generated from summing over terms which are not diagonal in the spectator particle. Coordinate space potentials are generally nonlocal when there is a momentum dependence of the interaction. In this case, the $\Gamma$ terms with complicated momentum dependence in Section \ref{sec:momentum} are the source.

From the 2PE term, a large number of unique terms are generated for the effective interaction. The $V_1$ piece again produces both a local part 
\begin{multline}
V^{1}_{12,\text{eff}}(\rot)=-\frac{c_1 m_\pi^6}{8\pi}\left(\frac{g_A}{F_\pi^2}\right)^2 \rho_{I=0}\taudot\:\\
 \times\left(\sqrt{\frac{8\pi}{15}}Y_2(\hat{r})\cdot\sigmatwo \w{2}{2}{2}{m_\pi r} -\frac{\sigmadot}{3}  \w{2}{2}{0}{m_\pi r} \right)
\end{multline}
and a nonlocal part 

\begin{multline}
V^{1}_{12,\text{eff}}(\rot,\rotp)=-\frac{c_1 m_\pi^9}{48\pi} \left(\frac{g_A}{F_\pi^2}\right)^2 
\left\{ \fracphantom \right. \\
\left[ \taudot \: \rhohat{0}{|\rot-\rotp|} + \tau_1^3 \rhohat{1}{|\rot-\rotp|} \right] \w{1}{1}{1}{ m_\pi r_{12}}\w{1}{1}{1}{ m_\pi | \rot-\rotp | } \\
\times \left( \sigmatwo \cdot [ Y_1 (\hat{r}_{12}) \otimes Y_1 \left(\frac{\rot-\rotp}{|\rot-\rotp|}\right) ]_2
 )  
 +\frac{\sigmadot}{3} Y_1 (\hat{r}_{12}) \cdot Y_1 \left(\frac{\rot-\rotp}{|\rot-\rotp|}\right)
+\rot\leftrightarrow\rotp\right) \\
%
+ \tau_1^3 \rhohat{1}{|\rot-\rotp|} \: \w{1}{1}{1}{m_\pi r_{12}} \w{1}{1}{1}{m_\pi | \rot-\rotp | } \\ 
\times \left(\fracphantom
\frac{1}{2}(\sigmacross)\cdot (Y_1 (\hat{r}_{12}) \times Y_1 \left(\frac{\rot-\rotp}{|\rot-\rotp|}\right) )+\rot\leftrightarrow\rotp\right) \\ 
%
- i \taucrossthree\: \rhohat{1}{|\rot-\rotp|} \: \w{1}{1}{1}{ m_\pi r_{12}}\w{1}{1}{1}{m_\pi | \rot-\rotp | }  \\
\times\left(\fracphantom
\frac{1}{2}(\sigmacross)\cdot (Y_1 (\hat{r}_{12}) \times Y_1 \left(\frac{\rot-\rotp}{|\rot-\rotp|}\right))-\rot\leftrightarrow\rotp\right) \\
%
+\left[3\rhohat{0}{|\rot-\rotp|}-\tau_1^3\rhohat{1}{|\rot-\rotp|}\right]\w{1}{1}{1}{ m_\pi r_{12}}w{1}{1}{1}{m_\pi \rotpr} \\
\times\left.\left(i\vec{\sigma}_1 \cdot (Y_1 (\hat{r}_{12}) \times Y_1(\rotphat)) + Y_1 (\hat{r}_{12}) \cdot Y_1(\rotphat) \fracphantom \right)
\right\}
\end{multline}

% 

For $V_3$ the spin and isospin structure of the 3N interaction is identical to that of $V_1$ so the result is similar. Again we find one term which is purely local,

\begin{multline}
V^{3}_{12,\text{eff}}(\rot)=\frac{c_3 }{4}\frac{m_\pi^6}{4\pi}\left(\frac{g_A}{F_\pi^2}\right)^2 \rho_{I=0}\;\taudot\:\\
 \times\left(-\sqrt{\frac{8\pi}{15}}Y_2(\hat{r})\cdot\sigmatwo\: \w{4}{2}{2}{m_\pi r} + \frac{\sigmadot}{3}\:  \w{4}{2}{0}{m_\pi r} \right)
\end{multline}

as well as a nonlocal part,
\begin{multline}
V^{3}_{12,\text{eff}}(\rot,\rotp)=\frac{c_3 }{4}\frac{m_\pi^9}{32\pi^2}\left(\frac{g_A}{F_\pi^2}\right)^2 \left\{\fracphantom\left[\taudot \:\rhohat{0}{|\rot-\rotp|} +\tauplusthree \rhohat{1}{|\rot-\rotp|} \right]\right. \\
%
\times\left(\sqrt{\frac{7}{12}}\frac{8\pi}{15}\sigmatwo\cdot  [ Y_2(\hat{r}_{12})\otimes Y_2 \left(\frac{\rot-\rotp}{|\rot-\rotp|}\right) ]_2 \w{2}{1}{2}{m_\pi r_{12}}\w{2}{1}{2}{m_\pi |\rot-\rotp|} 
\right.\\ 
%
+\frac{1}{3}\frac{8\pi}{15}\sigmadot Y_2(\hat{r}_{12})\cdot Y_2 \left(\frac{\rot-\rotp}{|\rot-\rotp|}\right)  \w{2}{1}{2}{m_\pi r_{12}}\w{2}{1}{2}{m_\pi |\rot-\rotp|} \\
%
-\frac{1}{3}\sqrt{\frac{8\pi}{15}}\sigmatwo \cdot Y_2(\hat{r}_{12}) \w{2}{1}{2}{m_\pi r_{12}}\w{2}{1}{0}{m_\pi |\rot-\rotp|} \\
-\frac{1}{3}\sqrt{\frac{8\pi}{15}}\sigmatwo \cdot Y_2 \left(\frac{\rot-\rotp}{|\rot-\rotp|}\right) \w{2}{1}{0}{m_\pi r_{12}}\w{2}{1}{2}{m_\pi |\rot-\rotp|} \\
\left.+\frac{1}{9}\sigmadot  \w{2}{1}{0}{m_\pi r_{12}}\w{2}{1}{0}{m_\pi |\rot-\rotp|}+\; \rot \leftrightarrow \rotp \fracphantom\right) \\
%
%
+\tau_1^3 \rhohat{1}{|\rot-\rotp|} \\
\times\frac{\sqrt{5}}{2}\left(\fracphantom  \sigmaone \cdot  [ Y_2(\hat{r}_{12})\otimes Y_2 \left(\frac{\rot-\rotp}{|\rot-\rotp|}\right) ]_1  \w{2}{1}{2}{m_\pi r_{12}}\w{2}{1}{2}{m_\pi |\rot - \rotp|} + \; \rot \leftrightarrow \rotp \fracphantom\right)\\
%
-i\taucrossthree \rhohat{1}{|\rot-\rotp|} \\
\times\frac{\sqrt{5}}{2}\left(\fracphantom \sigmaone \cdot  [ Y_2(\hat{r}_{12})\otimes Y_2 \left(\frac{\rot-\rotp}{|\rot-\rotp|}\right) ]_1  \w{2}{1}{2}{m_\pi r_{12}}\w{2}{1}{2}{m_\pi |\rot-\rotp|} - \; \rot \leftrightarrow \rotp \fracphantom\right) \\
%
%
+\left[3\rhohat{0}{|\rot-\rotp|}-\tau_1^3 \rhohat{1}{|\rot-\rotp|}\right]\left( \fracphantom\right. \\
\sqrt{\frac{5}{2}}\frac{8\pi}{15}  \vec{\sigma}_1 \cdot [Y_2 (\hat{r}_{12}) \otimes Y_2(\rotphat)]_1 \w{2}{1}{2}{m_\pi r_{12}} \w{2}{1}{2}{m_\pi\rotpr}  \\
-\frac{8\pi}{15}\;Y_2 (\hat{r}_{12}) \cdot Y_2(\rotphat) \;\w{2}{1}{2}{m_\pi r_{12}} \w{2}{1}{2}{m_\pi\rotpr} \\
%
-\frac{1}{3} \w{2}{1}{0}{m_\pi r_{12}} \w{2}{1}{0}{m_\pi\rotpr} \left. \left.\fracphantom\right)\right\}
\end{multline}

The $V_4$ term generates only a nonlocal term because the terms diagonal in the spectator particle all sum to zero. The nonlocal terms are

\begin{multline}
V^{4}_{12,\text{eff}}(\rot,\rotp) = \frac{c_4}{4}\frac{g_A^2}{F_\pi^2}\frac{m_\pi^9}{16\pi^2}\left\{ \fracphantom \right. 
\left(\rhohat{0}{|\rot-\rotp|}\taudot-\rhohat{1}{|\rot-\rotp|}\tauplusthree\right)
\\
\times\left( -\frac{4}{3}\sqrt{\frac{8\pi}{15}}\sigmatwo\cdot Y_2(\rotphat)\w{2}{1}{2}{m_\pi \rotpr}\w{2}{1}{0}{m_\pi |\rot-\rotp|} \right. \\
+\frac{2}{9}\sigmadot\; \w{2}{1}{0}{m_\pi \rotpr}\w{2}{1}{0}{m_\pi |\rot-\rotp|} \\
+\sqrt{\frac{7}{12}}\frac{8\pi}{15}\sigmatwo\cdot \left[ Y_2(\rotphat)\otimes Y_2 \left(\frac{\rot-\rotp}{|\rot-\rotp|}\right) \right]_2 \w{2}{1}{2}{m_\pi \rotpr}\w{2}{1}{2}{m_\pi |\rot-\rotp|} \\
-\frac{1}{3}\frac{8\pi}{15}\sigmadot \; Y_2(\rotphat)\cdot Y_2\left(\frac{\rot-\rotp}{|\rot-\rotp|}\right) \w{2}{1}{2}{m_\pi \rotpr}\w{2}{1}{2}{m_\pi |\rot-\rotp|} \\
-\frac{1}{3}\sqrt{\frac{8\pi}{15}}\sigmatwo\cdot Y_2\left(\frac{\rot-\rotp}{|\rot-\rotp|}\right)\w{2}{1}{0}{m_\pi \rotpr}\w{2}{1}{2}{m_\pi |\rot-\rotp|} \\
\left.+\; \rot \leftrightarrow \rotp \fracphantom\right) \\
+\left(\rhohat{0}{|\rot-\rotp|}\taudot-\rhohat{1}{|\rot-\rotp|}\tauplusthree\right) \\
\times \left( 
-\frac{\sqrt{21}}{3}\frac{8\pi}{15}\sigmatwo \cdot \left[Y_2(\hat{r}_{12})\otimes Y_2(\rotphat)\right]_2 \w{2}{1}{2}{m_\pi r_{12}} \w{2}{1}{2}{m_\pi \rotpr} \right. \\
-\frac{1}{3}\frac{8\pi}{15}\sigmadot\; Y_2(\hat{r}_{12}) \cdot Y_2(\rotphat)\w{2}{1}{2}{m_\pi r_{12}} \w{2}{1}{2}{m_\pi \rotpr} \\
+\frac{2}{9}\sigmadot \;\w{2}{1}{0}{m_\pi r_{12}} \w{2}{1}{0}{m_\pi \rotpr} \\
+\frac{1}{3}\sqrt{\frac{8\pi}{15}} \sigmatwo \cdot Y_2(\hat{r}_{12}) \w{2}{1}{2}{m_\pi r_{12}} \w{2}{1}{0}{m_\pi \rotpr} \\
+\frac{1}{3}\sqrt{\frac{8\pi}{15}} \sigmatwo \cdot Y_2(\rotphat) \w{2}{1}{0}{m_\pi r_{12}} \w{2}{1}{2}{m_\pi \rotpr} \left.\fracphantom\right) \\
%
-\left(\rhohat{0}{|\rot-\rotp|}\taudot-\rhohat{1}{|\rot-\rotp|}\tau_1^3 \right) \\
\times\sqrt{\frac{5}{2}} \frac{8\pi}{15} \vec{\sigma}_1\cdot\left[Y_2(\hat{r}_{12})\otimes Y_2(\rotphat)\right]_1 \w{2}{1}{2}{m_\pi r_{12}} \w{2}{1}{2}{m_\pi \rotpr} \left.\fracphantom\right\}
\end{multline}

\section{\label{sec:local}A first order local expansion}

Although nonlocality is a common feature of nuclear potentials \cite{PhysRevC.53.R1483}, working with local potentials is typically simpler and less computationally expensive. In light of this, we develop here a local approximation to the full density-dependent effective interaction.

The nonlocal parts can be systematically Taylor expanded in the coordinate $m_\pi(\vec{r}-\vec{r}\:')/2$ to give a series of purely local terms. To estimate the size of this expansion parameter, consider the naive model of an isospin symmetric nucleus with a spherical potential well. The size of the expansion parameter is bounded by the radius of the nucleus, so that 
\begin{equation}
m_\pi(\vec{r}-\vec{r}\:')/2 \lesssim 
\end{equation}

\appendix

\section{\label{app:momSums} Momentum space summation over $k_\delta$}

In this section we give explicit expressions for the sum over spectator momentum $\vec{k}_\delta$ in the momentum space expressions of Section~\ref{sec:momentum} assuming the zero-temperature Fermi-Dirac density of states. For $V_D$, we have three distinct integrals  of the form

\begin{equation}
\Gamma_{\alpha,N,P}(k) = \frac{k^{2-\alpha}}{m_\pi^3}\int^{|k_\delta|<k_F^{N,P}}\frac{d^3\vec{k}_\delta}{(2\pi)^3}  \frac{\left\{1,\vec{k}_\delta\cdot\hat{k},k_\delta^2\right\}_\alpha}{(\vec{k}-\vec{k}_\delta)^2+m_\pi^2}
\end{equation}

where $\left\{1,k_\delta\cos\theta,k_\delta^2\right\}_\alpha$ indicates that the numerator is $1$ for $\alpha=0$, $k_\delta\cos\theta$ when $\alpha=1$, etc.

The results of performing the integration are (suppressing the $N,P$ index for simplicity)
 \begin{align}
 \begin{split}
 \Gamma_{0}(k)&=\frac{k^2}{m_\pi^3}\int^{|k_\delta|<k_F}\frac{d^3\vec{k}_\delta}{(2\pi)^3} \frac{1}{(\vec{k}-\vec{k}_\delta)^2+m_\pi^2} \\
 &=\frac{3}{4}\frac{\rho}{m_\pi^3}\left[  \tilde{k}^2 -\tilde{m}_\pi\tilde{k}^2\left(\arctan\frac{1-\tilde{k}}{\tilde{m}_\pi}+\arctan\frac{1+\tilde{k}}{\tilde{m}_\pi}\right)\right. \\
 &\left.\qquad\qquad\qquad\qquad\qquad+\frac{1}{4}\left(1-\tilde{k}^2+\tilde{m}_\pi^2\right)\log\frac{\tilde{m}_\pi^2+(\tilde{k}+1)^2}{\tilde{m}_\pi^2+(\tilde{k}-1)^2} \right]
 \end{split} \\
 %
  \begin{split}
 \Gamma_{1}(k)&=\frac{k}{m_\pi^3}\int^{|k_\delta|<k_F}\frac{d^3\vec{k}_\delta}{(2\pi)^3} \frac{\vec{k}_\delta\cdot\hat{k}}{(\vec{k}-\vec{k}_\delta)^2+m_\pi^2} \\
 &=\frac{3}{4}\frac{\rho}{m_\pi^3}\left[  (3\tilde{k}^2-\tilde{m}_\pi^2-1)-4\tilde{k}^2\tilde{m}_\pi\left(\arctan\frac{1-\tilde{k}}{\tilde{m}_\pi}+\arctan\frac{1+\tilde{k}}{\tilde{m}_\pi}\right)\right. \\
 &\left.\qquad\qquad\qquad\qquad\qquad+\frac{(1+\tilde{m}_\pi^2)^2-3\tilde{k}^4+2\tilde{k}^2(1+3\tilde{m}_\pi^2)}{4\tilde{k}}\log\frac{\tilde{m}_\pi^2+(\tilde{k}+1)^2}{\tilde{m}_\pi^2+(\tilde{k}-1)^2} \right]
 \end{split} \\
 %
  \begin{split}
 \Gamma_{2}(k)&=\frac{k}{m_\pi^3}\int^{|k_\delta|<k_F}\frac{d^3\vec{k}_\delta}{(2\pi)^3} \frac{k_\delta^2}{(\vec{k}-\vec{k}_\delta)^2+m_\pi^2} \\
 &=\frac{1}{2}\frac{\rho}{m_\pi^3}\left[  (3\tilde{k}^2-9\tilde{m}_\pi^2-1)-6\tilde{m}_\pi(\tilde{k}^2-\tilde{m}_\pi^2)\left(\arctan\frac{1-\tilde{k}}{\tilde{m}_\pi}+\arctan\frac{1+\tilde{k}}{\tilde{m}_\pi}\right)\right. \\
 &\left.\qquad\qquad\qquad\qquad\qquad+\frac{3\left(\tilde{k}^4+\tilde{m}_\pi^4-6\tilde{k}^2\tilde{m}_\pi^2-1\right)}{4\tilde{k}}\log\frac{\tilde{m}_\pi^2+(\tilde{k}+1)^2}{\tilde{m}_\pi^2+(\tilde{k}-1)^2} \right]
 \end{split} 
  \end{align}
All variables appear in the dimensionless combinations $\tilde{m}_\pi=m_\pi/k_F^{N,P}$ and $\tilde{k}=k/k_F^{N,P}$. For actual calculations, one can replace $k_F$ with the observable densities using \eqref{eq:kF}.


\section{\label{app:wNotation}Fourier integrals}

Similar to \cite{PhysRevC.85.024003}, we introduce the following notation for the dimensionless radial functions $\w{\alpha}{\beta}{\ell}{ m r }$, defined by

\begin{align}
i^\ell \, Y_\ell(\hat{r}) \, \w{\alpha}{\beta}{\ell}{ m r}  &= \frac{4 \pi}{m^{3+\alpha-2\beta} }  \int \frac{d^3 q}{(2\pi)^3} \frac{q^\alpha}{(q^2+m^2)^\beta} Y_\ell(\hat{q}) \, e^{i \mathbf{q} \cdot \mathbf{r}  },
\end{align}
which implies that
\begin{align}\label{eq:wDef}
\w{\alpha}{\beta}{\ell}{z} =  \frac{2}{\pi} \int dk \, k^2 \, \frac{k^\alpha}{(1+k^2)^\beta} j_\ell(k z)
\end{align}
with $\vec{z} \equiv m \vec{r}$ and $k$ both dimensionless. In Table~\ref{table:wTable} we give the integrated results for selected values of $\alpha, \beta, \ell$. The first function we recover, for the case of $(\alpha, \beta, \ell ) = (0,1,0)$, is nothing but a Yukawa potential.

\begin{table}
\begin{center}
\begin{tabular}{| c c c | c |}
\hline
$\alpha$ & $\beta$ & $\ell$ & $ \w{\alpha}{\beta}{\ell}{z}$ \\
\hline
0 & 0 & 0 & $ 4\pi \delta^{(3)}(\vec{z}) $ \\
0 & 1 & 0 & $ \frac{\displaystyle e^{-z}}{\displaystyle z} $ \\
2 & 1 & 0 & $ -\frac{\displaystyle e^{-z}}{\displaystyle z} + 4\pi \delta^{(3)}(\vec{z}) $ \\
1 & 1 & 1 &  $\frac{\displaystyle e^{-z}}{\displaystyle z} (1 + \tfrac{1}{z})  $ \\
2 & 1 & 2 & $ \frac{\displaystyle e^{-z}}{\displaystyle z} (1+ \tfrac{3}{z} + \tfrac{3}{z^2} )$ \\
0 & 2 & 0 & $ \frac{\displaystyle e^{-z}}{2} $ \\
2 & 2 & 0 & $ \frac{\displaystyle e^{-z}}{\displaystyle z} (-\tfrac{z}{2} + 1 ) $ \\
2 & 2 & 2 & $  \frac{\displaystyle e^{-z}}{\displaystyle z} (\tfrac{z}{2} + \tfrac{1}{2} ) $ \\
4 & 2 & 0 & $  \frac{\displaystyle e^{-z}}{\displaystyle z} (\tfrac{z}{2} - 2 ) + 4 \pi \delta^3( \vec{z} )  $ \\
4 & 2 & 2 & $  \frac{\displaystyle e^{-z}}{\displaystyle z} ( - \tfrac{z}{2} + \tfrac{1}{2} + \tfrac{3}{z} + \tfrac{3}{z^2} ) $ \\
\hline
\end{tabular}
\end{center}
\caption{\label{table:wTable} Table of selected values for $\w{\alpha}{\beta}{\ell}{z}$ found by integrating equation \eqref{eq:wDef}.}
\end{table}

To further demonstrate the use of this integral, we now derive the familiar tensor force resulting from one pion exchange. This amplitude is related to the integral
\begin{equation} 
V_\pi( \mathbf{r}) =  -\frac{g_A^2 }{4 F_\pi^2} \:\taudot \int \frac{d^3 q } {(2 \pi)^3 } \frac{(\sigma^{(1)} \cdot \mathbf{q} ) (\sigma^{(2)} \cdot \mathbf{q} ) } {q^2+m_\pi^2} e^{i \mathbf{q} \cdot \mathbf{r} }. 
\end{equation}
We can rewrite the numerator in the integral as
\begin{equation} (\sigma^{(1)} \cdot \mathbf{q}) ( \sigma^{(2)} \cdot \mathbf{q} )= q^2 \left(\sqrt{\frac{8\pi}{15}} [\sigma^{(1)} \otimes \sigma^{(2)}]_2 \cdot Y_2(\hat{q}) + \tfrac{1}{3} \sigma^{(1)} \cdot \sigma^{(2)}  Y_0(\hat{q}) \right),\end{equation}
which then gives
\begin{equation} 
V_\pi( \mathbf{r}) = - \frac{g_A^2 }{4 F_\pi^2} \taudot \left(\frac{m_\pi^3}{4 \pi}\right) \left\{ - \sqrt{\frac{8\pi}{15}} [\sigma_1 \otimes \sigma_2]_2 \cdot Y_2(\hat{r}) \w{2}{1}{2}{m_\pi r} + \frac{1}{3} \sigma_1 \cdot \sigma_2 \w{2}{1}{0}{m_\pi r } \right\},  \end{equation}
or, plugging in explicit values for the $w$'s from Table \ref{table:wTable} and recalling that the tensor operator $S_{12}$ is conventionally defined
\begin{equation}
S_{12} \equiv   3\vec{\sigma}_1\cdot \hat{r}\vec{\sigma}_2\cdot \hat{r} -\sigmadot = 3\:\sqrt{\frac{8\pi}{15}} \:[\sigma_1 \otimes \sigma_2]_2 \cdot Y_2(\hat{r}) , 
\end{equation}
we have
\begin{equation} V_\pi( \mathbf{r})= \frac{g_A^2 }{4 F_\pi^2}  \left(\frac{m_\pi^3}{12 \pi}\right) \left\{ \frac{e^{-m_\pi r}}{m_\pi r} \left[ S_{12}\left(1+\frac{3}{m_\pi r}+ \frac{3}{m_\pi^2 r^2} \right) +  \sigma_1 \cdot \sigma_2 \right] - 4\pi  \sigma_1 \cdot \sigma_2 \delta^3(m_\pi \mathbf{r}) \right\}, 
\end{equation}
which is the familiar result.

Representations in terms of these functions are not unique. When $\alpha=2\beta$, one may rewrite the numerator in \eqref{eq:wDef} using only powers of $q$ less than $\alpha$. As a concrete example, consider the case for $\w{2}{1}{0}{z}$. Because
\begin{equation}
\frac{q^2}{q^2+m^2}=1-\frac{m^2}{q^2+m^2}
\end{equation}
we immediately see that $\w{2}{1}{0}{z}=\w{0}{0}{0}{z}-\w{0}{1}{0}{z}$, a result which can be confirmed by inspection from Table~\ref{table:wTable}. This rewriting also makes clear the origin of the short-range delta function terms in the Fourier transforms of the form $\w{2\beta}{\beta}{0}{z}$. Other relationships include,
\begin{align}
\w{4}{2}{0}{z}&=\w{0}{0}{0}{z}-2 \w{2}{2}{0}{z}-\w{0}{2}{0}{z} \\
\w{4}{2}{2}{z}&=\w{2}{1}{2}{z}-\w{2}{2}{2}{z}
\end{align}
When $\alpha < 2\beta$ similar relationships exist, but require introducing terms with higher powers of $q$ than occur in the original Fourier transform. The transform does not exist when $\alpha>2\beta$. In general, we have attempted to minimize the number of $w$ functions which appear in the expressions throughout this paper rather than expand them in this way.

We assume here that matrix elements will be evaluated within a formalism which regulates the interaction self-consistently, such as Harmonic-Oscillator-Based Effective Theory (HOBET) \cite{PhysRevC.77.034005}. If a momentum space regulator is desired, it may be naturally inserted into the Fourier transform \eqref{eq:wDef} thereby modifying the resulting expressions for the functions $w$. DO I NEED TO THINK ABOUT THE $k_\delta$ INTEGRAL?

Past work on the in-medium density dependent effective interactions has variously neglected the Dirac delta contact terms which arise from the Fourier transforms. WRITE SOME MORE HERE.


\bibliography{References}
\end{document}
%
% ****** End of file apssamp.tex ******
